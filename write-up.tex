\documentclass[a4paper]{article}
\usepackage{setspace}
\usepackage{amsmath}

\begin{document}
\onehalfspacing
\begin{center}
\textbf{Bugsy}\\
\textit{Michael and Andy}\\
\end{center}
\vspace{0.15in}

\section{Goal}
\section{Modules}
\subsection*{Bugsy}

Bugsy uses a utility function \begin{align*}
\hat{U} = -\big( w_f \cdot g^* (s) + w_t \cdot d \cdot delay \cdot t_{exp} \big)
\end{align*}
to prioritize its search, where $w_f, \ w_t$ are weights assigned by the user, $g^*$ is the cost function of the current state's path, $d$ is a distance metric, $delay$ is the estimated time between instantiation and expansion, and $t_{exp}$ is the estimated time to expand a node.

The utilities of all states in the exploration space are updated and re-ordered after a power of two expansion count has been reached, otherwise the last re-ordering is used to pop off a heap.

The key is understanding how the utility function helps the search is dynamic programming.
\end{document}